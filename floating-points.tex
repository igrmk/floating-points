\documentclass[english]{article}
\usepackage[T1]{fontenc}
\usepackage{amsmath}
\usepackage{amssymb}
\usepackage{babel}
\usepackage{enumitem}
\usepackage{stmaryrd}
\usepackage{cases}
\usepackage{amsthm}
\usepackage{numprint}

\newcommand{\idiv}{\backslash}
\newcommand{\llb}{\llbracket}
\newcommand{\rrb}{\rrbracket}

\renewcommand{\leq}{\leqslant}
\renewcommand{\geq}{\geqslant}
\renewcommand\qedsymbol{$\blacksquare$}

\newtheorem{lemma}{Lemma}
\newtheorem{example}{Example}

\begin{document}

\section{Notation}

\begin{description}[leftmargin=!, labelwidth=1cm]
    \item [$\mathbb{N}$]                  Integers
    \item [$\mathbb{N}^0$]                Non-negative integers
    \item [$\mathbb{N}^+$]                Positive integers
    \item [$\lfloor a \rfloor$]           Rounding towards zero, floor
    \item [$a \idiv b$]                   Integer division, the same as $\left\lfloor a/b \right\rfloor$
    \item [$b \mid a$]                    $b$ divides $a$ and $a$ is not zero
    \item [$b \nmid a$]                   $b$ doesn't divide $a$ or $a$ is zero
    \item [{$\left[a \right]_N$}]         Rounding to nearest $10^{-N}k, k \in \mathbb{N}$, away from zero in case of uncertainty
    \item [{$\left\llb a \right\rrb_N$}]  Rounding $N$ significant decimal figures, away from zero in case of uncertainty
\end{description}

\section{Binary to decimal conversion}

Let us have \emph{positive} binary floating-point number
\begin{equation}
    x = 2^{a}p,\: p \in \mathbb{N}^0, a \in \mathbb{N}
\end{equation}

We need to find decimal mantissa $m$ and decimal exponent $e$ such that
\begin{align*}
    x & = 10^e \times m \\
    m & \in \mathbb{N}^0 \\
    10 & \nmid m \\
    e & \in \mathbb{N}
\end{align*}

It is always possible to rewrite it in such a way
that mantissa is not divisible by neither $2$ nor $5$.
\begin{align}
    x & = 2^b 5^d q \label{canon} \\
    q & \in \mathbb{N}^0 \nonumber \\
    2 & \nmid q \nonumber \\
    5 & \nmid q \nonumber \\
    p & = 2^{b-a} 5^d q \nonumber
\end{align}

Taking into account that $5^d = 10^d 2^{-d}$ we can rewrite it extracting tens exponent.
\begin{align}
    x & = 10^d \times 2^{b-d} q \\
    x & = 10^b \times 5^{d-b}q
\end{align}

The mantissa and the exponent should be integers.
Therefore final result for decimal representation of given floating-point number is
\begin{align*}
    (m, e) & = \begin{cases}
        (2^{b-d} q, d), & b-d \geq 0 \\
        (5^{d-b} q, b), & b-d < 0
    \end{cases} \\
    m & = 2^{\max(b-d, 0)} 5^{\max(d-b, 0)} q \\
    e & = \min(b, d)
\end{align*}

\begin{example}
For double-precision floating-point numbers following inequations are true
    \begin{align*}
        0 & \leq d \leq 22 \\
        \numprint{-1022} & \leq b \leq \numprint{1075} \\
        0 & \leq q \leq 2^{53} - 1 = \numprint{9007199254740991} \\
        \numprint{-1022} & \leq e \leq 22 \\
        0 & \leq m \leq (2^{53} - 1) \times 5^{1022}
    \end{align*}
\end{example}

\section{Binary to decimal conversion with rounding to $N$ decimal digits after point}

Firstly let us define rounding meaning here, we denote rounded number as $\left[x \right]_{N}$.
We define rounding on $k$th interval where $k \in \mathbb{N}^0$.
\begin{align*}
    \left[x \right]_N = \frac{k}{10^N},\,
    \forall x \in \begin{cases}
        \left[\frac{k-0.5}{10^N}, \frac{k+0.5}{10^N} \right), & k>0 \\
        \left[0, \frac{0.5}{10^N} \right), & k=0
    \end{cases}
\end{align*}

We will use following known trick to compute rounded value.
\[
    \left[x \right]_N = \frac{\left\lfloor 10^N x + 0.5 \right\rfloor}{10^N} \\
\]

We can consider only case $N < -e$.
Otherwise rounded number will be equal to a given one.
Let us denote $x' = \left\lfloor 10^N x + 0.5 \right\rfloor$
and rewrite it in terms of the powers from equation \eqref{canon}.
\begin{align}
    x' & = \left\lfloor 5^{d+N} 2^{b+N}q + 0.5 \right\rfloor \nonumber \\
    d' & = d+N \nonumber \\
    b' & = b+N \nonumber \\
    x' & = \left\lfloor 5^{d'} 2^{b'} q + 0.5 \right\rfloor \label{x-prime}
\end{align}

Finally this leads us to
\begin{numcases}{x'=}
    5^{d'} 2^{b'} q, &
    $d' \geq 0, b' \geq 0$
    \label{round-to-digits-gg}
    \\
    \left(5^{d'} q + 2^{-b'-1} \right) \idiv 2^{-b'}, &
    $d' \geq 0, b' < 0$
    \label{round-to-digits-gl}
    \\
    \left(2^{b'} q + 5^{-d'} \idiv 2 \right) \idiv 5^{-d'}, &
    $d' < 0, b' \geq 0$
    \label{round-to-digits-lg}
    \\
    \left(q + 5^{-d'} 2^{-b'-1} \right) \idiv \left(5^{-d'} 2^{-b'} \right), &
    $d' < 0, b' < 0$
    \label{round-to-digits-ll}
\end{numcases}

Cases
\eqref{round-to-digits-gg},
\eqref{round-to-digits-gl},
\eqref{round-to-digits-ll}
obviously follow from \eqref{x-prime}. Let's show that \eqref{round-to-digits-lg} is correct.

\begin{lemma}
    The following assertion is correct for every $x \in \mathbb{N}^0$, $y \in \mathbb{N}^0$ and $q \in \mathbb{N}^0$
    \[
        \left\lfloor \frac{2^x q}{5^y} + \frac{1}{2} \right\rfloor = \\
        \left\lfloor \frac{2^x q - \frac{1}{2}}{5^y} + \frac{1}{2} \right\rfloor
    \]
\end{lemma}

\begin{proof}
    It is enough to proof that $\frac{2^x q}{5^y} + \frac{1}{2} < n+1$
    follows to $\frac{2^x q - \frac{1}{2}}{5^y} + \frac{1}{2} < n+1$
    and $\frac{2^x q}{5^y} + \frac{1}{2} \geq n$
    follows to $\frac{2^x q - \frac{1}{2}}{5^y} + \frac{1}{2} \geq n$
    where $n \in \mathbb{N}^0$.
    Former is obvious. Let us proof latter. At first let us rewrite inequation.
    \begin{align*}
        \frac{2^x q}{5^y} + \frac{1}{2} & \geq n \\
        \frac{5^y}{2} & \geq 5^y n - 2^x q
    \end{align*}

    Note that $2 \nmid 5^{y}$ and right hand side is integer, this leads us to
    \begin{align*}
        \frac{5^y - 1}{2} \geq 5^y n - 2^x q \\
        \frac{2^x q - \frac{1}{2}}{5^y} + \frac{1}{2} \geq n
    \end{align*}
\end{proof}

\section{Rounding to $F$ significant figures}

Rounded mantissa $m''$ should satisfy following inequation.
\[
    0 \leq m'' < 10^F
\]

Therefore we need to find minimal $g \in \mathbb{N}^0$ such that
\begin{align*}
    m / 10^g & < 10^F - 0.5 \\
    \therefore 2m / 10^g & < 2 \times 10^F - 1 \\
    \therefore (2m) \idiv 10^g & < 2 \times 10^F - 1
\end{align*}

Now we can find $g$ iteratively or use faster method if $g$ is known to be big.
Then we round a value to $e-g$ digits.
\end{document}
